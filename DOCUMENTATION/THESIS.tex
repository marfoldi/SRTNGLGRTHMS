\documentclass{elteikthesis}

\usepackage{url}
\usepackage{fancyhdr}
\usepackage{indentfirst}
\usepackage{enumitem}
\usepackage{ucs}
\usepackage[utf8]{inputenc}
\usepackage[T1]{fontenc}
\usepackage[english,hungarian]{babel}
\selectlanguage{hungarian}

\title{Rendezési algoritmusok szemléltetése}
\author{Márföldi Péter Bence}
\supervisor{Veszprémi Anna}
\supervisorstitle{mestertanár}
\period{programtervező informatikus BSc}
\thesisyear{2015}
\department{Algoritmusok és Alkalmazásaik Tanszék}

\fancyhf{}
\fancyhead[L]{\rightmark}


\begin{document}

\frontmatter

	\maketitle
	\tableofcontents
	
\mainmatter

	\cfoot{\thepage}
	\pagestyle{fancy}
	\setlist[description]{leftmargin=\parindent,labelindent=\parindent,itemsep=1.5pt}

\chapter{Bevezetés} 
Az bizonyos, hogy minden informatikus - beleértve a leendőeket is - tanulmányaik kezdetén találkoztak a rendezési algoritmusokkal. Nagyszerű terület arra, hogy megérthessük a műveletigény kérdését, azt hogy mi számít igazán sok adatnak, vagy éppen, hogy mit értünk egy algoritmus stabilitásán.

\section{A feladat és annak értelmezése} 

\section{Alkalmazott technológiák}
A Következőkben röviden összefoglaljuk a Java\cite{Java} és a JavaFX\cite{JavaFX} jellegzetességeit.
\subsection{A Java-ról röviden}
A Java egy általános célú, objektumorientált programozási nyelv, melyet 2009-ig a \emph{Sun Microsystems} fejlesztett, ezt követően pedig az \emph{Oracle}. A szakdolgozatban használt 1.8-as verziót már az \emph{Oracle} adta ki 2014-ben. A Java nyelv a szintaxisát a C és C++ nyelvektől örökölte, azonban utóbbitól eltérően egyszerű objektummodellel rendelkezik. \par
 A Java platformra készült programok túlnyomó többsége asztali alkalmazás. Manapság egyre több helyen találkozhatunk a Java nyelven írt programokkal, például mobil eszközökön, banki rendszereknél vagy akár egy szórakoztató elektronikai eszközön. Nagy előnye, hogy sok nyelvvel ellentétben platformfüggetlen, azaz egy adott platformról egy program minimális változtatással átültethető egy másik platformra. \par
 A Java legfontosabb része a \emph{Java virtuális gép (JVM)}. A \emph{JVM}-et sokféle berendezés és szoftvercsomag tartalmazza, így a nyelv egyaránt platformként és középszintként is működik.
Összefoglalva a Java program három fontos szerepet tölt be:
\begin{description}
\item[$\bullet$] programozási nyelv
\item[$\bullet$] köztes réteg (middleware)
\item[$\bullet$] platform
\end{description}

\subsection{JavaFX}
Olyan szoftverplatform, amelynek célja, hogy gazdag internetes alkalmazást lehessen készíteni és futtatni eszközök széles skáláján. Eredetileg a \emph{Swing} könyvtárat váltotta volna fel, azonban jelenleg mindkettő része a \emph{Jave SE}-nek. \par
 A 2.0-ás verzióig a fejlesztők egy külön nyelvet használtak, amelyet \emph{JavaFX Script}-nek neveznek. Azonban mivel ez szintén Java bájtkódot generál a későbbiekben megadatott a lehetőség, hogy a programozók Java kódot használjanak helyette. A JavaFX egyik legnagyobb előnye, hogy egy egyszerű \emph{XML} struktúrában leírhatók a program grafikus felületének összetevői, melyhez ezt követően elegendő az egyes interakciókhoz tartozó funkciókat implementálni.\par Az elterjedtebb operációs rendszerek mindegyikét támogatja. Ahogyan előnye, úgy hátránya is a \emph{Swing}-hez képest az, hogy jelenleg is folyik a fejlesztése, ezért olykor csak hosszas utánajárást követően sikerül megoldást találni egy-egy problémára.

\chapter{Felhasználói dokumentáció}

\section{A vizsgált algoritmusok}

\subsection{Buborékrendezés}
A legrégebbi és a legegyszerűbb rendezési algoritmus. Mindemellett a legtöbb esetben a leglassabb is. Már az 1965-ös évben megjelent egy teljes körű elemzése\cite{Demuth}.\par
 A rendezés minden egyes elemet összehasonlít a rákövetkező elemmel, és ha szükséges megcseréli őket. Mindezt addig, amíg nincs egy olyan menet, amelyben egyetlen elem sem cserél helyet. Ez azt eredményezi, hogy lépésenként a maximális elem "buborék" szerűen a lista végére kerül, ezzel egyidejűleg a kisebb elemek "lesüllyednek" a tömb elejére. Az algoritmus javítható azzal, hogy nem vizsgáljuk meg minden menetben a tömb összes elemét, hanem amennyiben egy maximális elem elérte a helyét visszavezetjük a problémát az eggyel "rövidebb" rendezési feladatra\cite{Fekete}.

\subsection{Beszúró rendezés}
A nevéből egyszerűen kikövetkeztethető az eljárás: beszúrja az elemeket a megfelelő helyükre a végleges tömbbe. Lényege, hogy a soron következő elemet egy ideiglenes változóba mentjük, és a rendezett tömb elemeit jobbra csúsztatjuk, mindaddig amíg a kiválasztott érték nem kerül a helyére. Kezdetben a tömb első elemét tekintjük rendezettnek. A legtöbb esetben akár kétszer hatékonyabb a Buborékrendezéshez képest\cite{Fekete}. Továbbá kis (néhány száz) elemszámú bemenetre az egyik leghatékonyabb algoritmus.

\subsection{Shell rendezés}
Donald Shell nevéhez fűződik, a legtöbb esetben a leggyorsabb négyzetes idejű algoritmus. Többször vizsgálja a tömböt, és minden alkalommal egy részén beszúró rendezést hajt végre. Arra, hogy mekkora méretű résztömböt vizsgáljon az egyes lépésekben az algoritmus több javaslat is található. A teljesség igénye nélkül néhány ajánlás\cite{ShellWiki} erre vonatkozóan:\par
\begin{table}[h]
	\def\arraystretch{2}
	\centering
	\begin{tabular}{|c|c|c|c|}
		\hline
		\textbf{Szabály (k=1...)} & \textbf{Konkrét értékek}  & \textbf{Legrosszabb eset} & \textbf{Szerző}         \\ \hline
		$\lfloor n / 2^k \rfloor$&      $\left\lfloor\frac{n}{2}\right\rfloor,
		\left\lfloor\frac{n}{4}\right\rfloor, \ldots$, 1            &              $\Theta(n^2)$                      & Shell, 1959     \\ \hline
		$2^p 3^q$ egymást követően & 1, 2, 3, 4, $\ldots$  &               $\Theta(n \log^2 n)$                     & Pratt, 1971     \\ \hline
		$\left\lceil \frac{9^k-4^k}{5\cdot4^{k-1}} \right\rceil$& 1, 4, 9, 20, $\ldots$ &             Nem ismert                       & Tokuda, 1992	\\ \hline
	\end{tabular}
\end{table}
 Az eredeti algoritmus időkomplexitása a legrosszabb esetben $\Theta(n^2)$. A fentebbi táblázatból látható, hogy az algoritmus sebessége nagyban függ a lépésköz megválasztásától. A program Vaughan Ronald Pratt által javasolt értékeket\cite{Pratt} használja, így legrosszabb esetben is $\Theta(n \log^2 n)$ a műveletek száma.

\subsection{Gyorsrendezés}
C.A.R. Hoare\cite{Horae} alkotta meg 1965-ben. Az egyik leggyorsabb rendezési eljárás, ezért rendkívül gyakran alkalmazzák.\par
Helyben rendező, oszd meg és uralkodj\cite{Cormen} elven működő rekurzív algoritmus. A következő négy lépésre bontható fel az rendezés:
\begin{description}
	\item[$\bullet$] Ha csak egy vagy nulla elemű az elemzett rész, akkor ne tegyünk semmit.
	\item[$\bullet$] Válasszunk egy vezérelemet (legjobb oldalibb elem).
	\item[$\bullet$] A rendezendő részt vágjuk ketté, az egyik oldalára a vezérelemtől kisebb, míg a másikra a nagyobb elemek kerüljenek.
	\item[$\bullet$] Rekurzívan ismételjük meg az előbbi lépéseket a résztömbökön.
\end{description}
A rendezés műveletigényét befolyásolja, hogy hogyan választjuk meg a vezérelemet. Például a legnagyobb műveletigényt ($\mathcal{O}(n^2)$) eredményezi, ha mindig a legjobboldalibb elemet választjuk vezérelemnek, és a tömb elemei csökkenő sorrendben vannak\cite{Cormen}. Éppen ezért a gyakorlatban javasolt ezen elem véletlenszerű megválasztása. Az egyszerűbb megérthetőséget szem előtt tartva a program mindig a legjobboldalibb elemet szelektálja.\par
 A gyorsrendezés a legtöbb esetben(közepes és nagy méretű bemenetre) a legmegfelelőbb választás ha számít a rendezés sebessége. Azonban ha a tömb elemei már eleve rendezettek vagy esetleg fordított sorrendben szerepelnek sajnos nem hatékony\cite{Cormen}. Ezekben az esetekben ajánlatos másik algoritmust használni.

\subsection{Kupacrendezés}
Az $\mathcal{O}(n\log n)$ algoritmusok közül az egyik leglassabb, előnye a gyorsrendezéssel szemben, hogy nem erőteljesen rekurzív. Ennél fogva jól alkalmazható milliós nagyságrendű bemenetre. Ahogyan a neve is sugallja, a rendezéshez egy kupac adatszerkezetet használ. Az algoritmus ismertetése előtt definiáljuk a kupac fogalmát\cite{Fekete}:
Olyan bináris fa, amelyre a következők teljesülnek:
\begin{description}
	\item[$\bullet$] Kizárólag a levelek szintjén hiányozhat csúcs, azaz "majdnem teljes".
	\item[$\bullet$] A levélszint csúcsai balra tömörítettek.
	\item[$\bullet$] Minden belső csúcs értéke nagyobb vagy egyenlő, mint a gyerekeinek értékei.
\end{description}
A második pont értelmében egyetlen olyan csúcs lehet, amelynek csak egy gyereke van, és az közvetlenül a levélszint felett kell, hogy elhelyezkedjen.\par
A rendezés az előbbi tulajdonságokra támaszkodik. 
Az algoritmus a bemeneti elemekből kupacot épít, majd a legfelsőbb elemét áthelyezi a kupac "végére". Ezt követően ellenőrzi a kupac tulajdonságokat, ahol szükséges cseréket hajt végre, hogy helyreálljon a kupac adatszerkezet, ekkor már az utolsó legjobboldalibb levélelem nem vesz részt a kupacépítésben. A gyökérben található elemet a legjobboldalibb levélelem elé helyezi, és újraépíti a kupacot. Ezen lépések addig ismétlődnek, amíg már a maximális elem áthelyezése nem lehetséges, a kupac "végére" helyezési művelet elérte a gyökeret.\par

\subsection{Versenyrendezés}
A maximum-kiválasztó rendezések közé tartozik, minden egyes menetben kiválasztja a legnagyobb elemet, kiírja és végül eltávolítja. A maximum kiválasztásnak a gyakorlati hátterét a sportesemények lebonyolítási rendje adja, azaz meghatározza az elemek között a "nyertest"\cite{Fekete}. A módszert n=2$^k$ inputhossz esetén érdemes alkalmazni, mivel ettől értérő bemenetre sokkal kedvezőbb eredményt lehet elérni a kupacrendezéssel\cite{Fekete}. \par
Az algoritmus által használt adatszerkezet egy teljes bináris fa. A bináris fa leveleiben szerepelnek a rendezendő elemek. Az első speciális menetben a fa belső pontjait kitöltjük, úgy, hogy a pontba a gyerekei közüli nagyobb érték kerül.\par 
Ezt követően kerül sor az $(n-1)$ egyszerűbb menetre: A gyökérben található elemet keresve "lefelé" haladunk a bináris fában, majd megtalálva azt a levelet amelyben a gyökér értéke szerepel egy abszolút vesztest állítunk a helyére. Ez az érték a programban -1, mivel csak pozitív egészeket használunk a rendezések szemléltetésére. Ezzel ellentétben a gyakorlatban ez az érték $-\infty$. Majd ezen az "ágon" újrajátsszuk a mérkőzéseket.
\par
A rendezés egyetlen hátránya a tárigénye, $n$ szám esetén további $n-1$ mezőre szükség van a versenyfa elkészítéséhez. Éppen emiatt a gyakorlatban nem sűrűn használt eljárás.

\subsection{Radix "előre"}
Az előzőekben ismertetett algoritmusok mindegyike összehasonlításon alapuló rendezés. A radix rendezés viszont az edényrendezések közé tartozik. Ezen rendezések nem hasonlítják össze az elemeket, hanem az elemek az értéküknek megfelelő edényekbe kerülnek. Az edényrendezések eredményeként rendezett adatsorozatot kapunk lineáris időben. A radix rendezés egy rekurzív algoritmus, melynek minden szintjén létrejönnek az edények.\par
Az általános edényrendezés egy speciális változata a radix előre rendezés, bináris, $d$ hosszú számokra.\par
Az első menetben a rendezendő elemek első bitjét vizsgálja az algoritmus. A vizsgálat két mutatóval történik, melyek a tömb két végéről indulnak. A tömb elején addig halad a mutató, amíg a vizsgált elem első jegye nem 1, ezzel párhuzamosan a tömb végén olyan elemet keres a másik, melynek első jegye 0. Amennyiben talált ilyen elemeket megcseréli őket. Ezt mindaddig folytatódik, amíg a két mutató nem találkozik. Ekkor kialakul két edény, az elsőben a 0-ás kezdőbittel rendelkező számok, míg a másodikban az 1-essel kezdődő elemek foglalnak helyet. Ezt követően a második bit kerül vizsgálatra az "aledényekben", az előzővel azonos módon. A rendezés befejeződött, ha minden számjegy szerinti vizsgálat megtörtént, vagy ha mindegyik, a futás alatt kialakult edény már csak egy elemet tartalmaz.

\subsection{Radix "vissza"}
Az előző algoritmustól eltérően már nem helyben rendez, az eljárásnak két tömbre van szüksége. További különbség, hogy a kisebb helyiértéktől a nagyobb felé halad a vizsgálat. Amennyiben az aktuálisan vizsgált bit értéke 0, akkor a "második" tömb elejére, ellenkező esetben a tömb végére töltjük át az aktuális bináris számot.

\chapter{Fejlesztői dokumentáció}

\section{Tervezés és megvalósítás}
A fejlesztés során több szempontot is figyelembe kell venni, úgy mint: műveletigény, memóriaigény, jó megjelenés, egyszerű kezelhetőség, és átlátható-, bővíthető kód készítése. Mivel ezen kritériumok közül több is csak egy másik rovására javítható, ezért a tervezés során kompromisszumokat kell kötni. Továbbá fel kell készülni arra, hogy az eredeti terven a fejlesztés során módosításokat kell végezni, mivel egy-egy probléma megoldása más megközelítést kívánhat.

\subsection{Tervezés}
A dolgozat fő célja egy olyan elsősorban hallgatóknak szánt program létrehozása, amellyel néhány rendezési algoritmus működése egy letisztult és egyszerű felhasználó felületen keresztül tanulmányozható.\par
A programnak három jól elkülönülő részből kell állnia:
Egy logikai(modell) részből, ami gyakorlatilag a rendszer "motorja", itt kell, hogy történjen mindenféle számítási és adattárolási művelet. Egy megjelenítési rétegből, amely a logikai rész eredményeit jeleníti meg a felhasználó számára. Végül pedig egy kontroller szintből, amely kapcsolatot teremt a logikai- és a megjelenítési réteg között. A gyakorlatban ezt a fajta tagolást nevezik Modell-Nézet-Vezérlő \emph{(MVC)} tervezési mintának.\par
Az elsődleges szempont az, hogy a felhasználó könnyedén tudja kezelni a programot, és segítségével megértse az algoritmusok működését. Így a felhasználói felület áttekinthetőségére és letisztultságára nagy hangsúlyt kell fektetni. Továbbá fontos az is, hogy a jövőben több rendezési eljárást is könnyedén meg lehessen jeleníteni a jelenlegiek mellett, így fontos szempont a kód egyszerű bővíthetősége.

\subsection{Megvalósítás}
Az első lépés a rendezési algoritmusok implementálása.
Ezt követhette egyszerűbb felhasználói felület létrehozása. Kezdetben elegendő, ha csak egy grafikon jelenik meg, amely reprezentálja a tömbben található számokat.\par
Két algoritmushoz szükséges a gráfos megjelenítés, így a következő lépés egy gráf implementálása.
Ezt követően a cél, hogy néhány "beégetett" elemre a rendezések lejátszhatóak legyenek, és az aktuális állapota a tömbnek szinkronban legyen a diagrammal valamint a gráffal. Később az egyes lépésekben történő összehasonlításokat/vizsgálatokat, mozgatásokat, cseréket kell különböző színekkel jelölni az állapotjelző felületeken és számon tartani ezen műveletek összegeit.\par
Ezen a ponton az módosítás történt a projekt tervein. Eredetileg egy-egy külön szálon futottak volna az algoritmusok, és a felhasználói interakció hatására ezek állapota változott volna. Azonban a \emph{JavaFX} szálkezelése jelentősen eltér az szokványos szálkezelésétől, ezért járhatóbb útnak bizonyult az, hogy kétszer kerüljenek implementálásra az algoritmusok.\par Az egyik implementációban elmentjük az interakciót követő állapotot, és ez jelenik meg a felhasználói felületen. A másik megvalósításban pedig a rendezések azonnal lezajlanak, így képet kaphatunk arról, hogy mennyi műveletre volt szükség az egyes eljárások során. Ezen utóbbi implementációk mindegyike külön szálon fut, és ahogy valamelyik befejeződik figyelmezteti a főprogramot, hogy jelenítse meg a műveletek számát. Valamint minden szálhoz egy-egy egyszerűbb egységteszteket kell készíteni.\par
Miután a program alapjai elkészültek kezdetét veheti a a felhasználói felület részletes kialakítása. Elsőként a diagram elhelyezése egy panelen, amely tartalmaz továbbá egy listát a választható algoritmusokról. Illetve egy táblázatot, melyben szerepelnek az aktuális állapot egyes tulajdonságai.\par
A programnak egy fontos szolgáltatása az, hogy a felhasználó különböző adatbeviteli mód közül választhat. A logikai réteget ki kell bővíteni ezekkel az esetekkel, továbbá a felhasználói felületen lehetőséget adni ezen módok kiválasztására.\par
Ezután az eszköztár kerül a helyére, mellyel párhuzamosan megtörténik az egyes műveletekhez tartozó eljárások implementálása.\par
Az utolsó teendő a felhasználói felületen egy rendezések összehasonlítására lehetőséget adó panel létrehozása, táblázattal, benne az algoritmusok műveletigényével. Továbbá egy diagrammal, amin megjelenik a táblázatból kiválasztott sor összehasonlításainak és mozgatásainak a száma.\par
Végül, hogy a program egyszerűen használható legyen \emph{Windows} környezetben egy telepítő fájl készítése, amellyel az előbb említett operációs rendszert használóknak nem szükséges külön \emph{Java}-t telepíteniük.


\subsection{Használt fejlesztőeszközök}
A fejlesztés \emph{Eclipse SDK 4.4} fejlesztői környezet keretei között történt. A program grafikus fejlesztői felületet ad alkalmazások készítéséhez.\par
A program elkészítése során a kódolást segítő funkció volt a kódkiegészítés, továbbá az egyik beépített projektmenedzsment eszköz(\emph{EGit}).\par
A fejlesztéshez elengedhetetlen a \emph{Java SE 8u40} vagy magasabb verziójú szoftver. Továbbá a fejlesztést nagyban elősegítette a \emph{JavaFX Scene Builder 2.0}, melynek segítségével egyszerűen megtervezhetővé váltak a grafikus felület komponensei.\par
Végül a telepítési környezet létrehozásához \emph{Ant} és \emph{InnoSetup} eszközök kerültek felhasználásra.
Az egész projekt, beleértve e dokumentumot is megtalálható, és az egyes verziók visszakövethetők a GitHub-on: \url{https://github.com/marfoldi/SRTNGLGRTHMS}\par
A program fejlesztése során egyedüli külső függvénykönyvtár a \emph{JUnit} volt, melynek segítségével egységtesztek készültek.

\begin{thebibliography}{widest entry}
\bibitem{Java}
\emph{Java (programming language)},
Wikipedia the free encyclopedia.
[ONLINE] [Hivatkozva: 2015.04.21]
\url{http://en.wikipedia.org/wiki/Java_(programming_language)/}
\bibitem{JavaFX}
\emph{JavaFX},
Wikipedia the free encyclopedia.
[ONLINE] [Hivatkozva: 2015.04.21]
\url{http://en.wikipedia.org/wiki/JavaFX/}
\bibitem{Demuth}
Demuth, H.:
\emph{Electronic Data Sorting},
PhD thesis, Stanford University,
1956, [184]
\bibitem{Fekete}
Dr. Fekete István:
\emph{Algoritmusok és adatszerkezetek I. jegyzet},
[ONLINE] [Hivatkozva: 2015.04.20] \url{http://people.inf.elte.hu/fekete/algoritmusok_bsc/alg_1_jegyzet/}
\bibitem{Cormen}
Thomas H. Cormen, Charles E. Leiserson,Ronald L. Rivest, Clifford Stein:
\emph{Új algoritmusok},
Scolar kiadó, 2003,
[992], 9789639193901
\bibitem{Horae}
C.A.R. Hoare:
\emph{Algorithm 64: Quicksort}
Communications of the ACM, 4, 7, 1961
\bibitem{ShellWiki}
\emph{Shellsort},
Wikipedia the free encyclopedia.
[ONLINE] [Hivatkozva: 2015.04.25]
\url{http://en.wikipedia.org/wiki/Shellsort/}
\bibitem{Pratt}
Vaughan Ronald Pratt:
\emph{Shellsort and Sorting Networks (Outstanding Dissertations in the Computer Sciences)}
Garland, 1980, [62], 0824044061 
\end{thebibliography}

\end{document}
